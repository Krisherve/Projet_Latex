\documentclass[12pt,a4paper]{article} % classe article, taille 12, papier A4
\usepackage{fontspec} % permet de gérer la police

\usepackage{graphicx} % gère les images
\usepackage{array} % gère les tableaux
\usepackage{hyperref} % gère les liens
\usepackage{csquotes} % gère les citations et les guillemets
\usepackage[utf8]{inputenc}


\usepackage{polyglossia} % gère l'aspect multilingue des documents
\setmainlanguage{french} % sélection de la langue principale du document

\usepackage[citestyle=verbose]{biblatex} % appel de biblatex, de nombreuses autres options disponibles

\addbibresource{gafam.bib}
\bibliography{gafam} % appel du fichier .bib sans l'extension

% emplacement pour d'autres packages

\title {\textbf {Les enjeux politiques de l’emprise des GAFAM sur l’espace public numérique }} % le titre du document
\author{\textit{NGAJEU DE NUI KRIS}} % son auteur
\date{} % la date


%% début du document

\begin{document} % début du corps du texte
	
	\maketitle % affiche le titre, l'auteur et la date
	
	
	\begin{figure}[h]
		\centering
		\includegraphics[scale=1]{ulb}
	\end{figure}
	
	
	\tableofcontents % affiche une table des matières correspondant aux sections du document
	
	\section{\underline{Introduction}}
	Pendant l’été 2018, les \textbf{GAFAM} \textit{(Google, Apple, Facebook, Amazon, Microsoft)} ont été les chefs de file de l’embellie boursière exceptionnelle de Wall Street. Ces sociétés connues dans le monde entier développent et utilisent des \textbf{systèmes d’exploitation , des équipements informatiques, des réseaux de télécommunication et des « data centers »}. Elles inventent chaque jour des utilisations commerciales sur Internet. Leur succès est généralement expliqué par leur capacité d’innovation, même s’il leur est trop facilement accordé la paternité de l’internet moderne.
	Pourtant, à l’origine, Internet était un bien public\footcite{Smyrnaios2017} , géré par \textbf{l’ARPANET} de \textbf{l’ARPA} \footnote[2]{Advanced Research Projects Agency}, dépendant du Département de la Défense des Etats-Unis, en vue de développer de nouvelles technologies militaires. Au tout début des années 1970\footnote[3]{Ray Tomlinson va créer le courrier électronique, destiné à renforcer la communication entre les informaticiens.} , l’idée fondamentale de ce projet consistait à développer:\begin{itemize}
		\item  une architecture décentralisée,
		\item le « time sharing »,
		\item et le travail autonome
	\end{itemize}    ce qui allait dans le sens contraire des technologies centralisatrices alors dominantes.
	Au milieu des années 1980, ARPA se désengage du projet, lequel est repris par la NSF\footnote[4]{la National Science Foundation}, qui crée une \textbf{ « république d’informaticiens »}, notamment les Universités et les centres de recherche publics, financée par le contribuable américain, en vue d’innover et d’imposer les normes américaines au monde. Cependant, même si le secteur paraissait économiquement très prometteur, fondamentalement l’idée était d’offrir les innovations gratuitement et librement dans les réseaux, d’interdire le dépôt de brevets sur les inventions (jugées collectives) et d’empêcher l’appropriation de ce bien jugé, par nature, public. Cet effort aboutira à la conception et la mise en place du fameux (www) \footnote[5]{World Wide Web}. 
	Cependant, le processus de \textit{dérèglementation (absence de régulateur privé et public)} décidé par le Président \textbf{Ronald Reagan} a rapidement attisé les espoirs et les opportunismes marchands, notamment ceux des firmes privée de l’économie numérique. Internet présentait dès l’origine un intérêt économique considérable, car la demande croissante de ses services permet l’expression de rendements croissants conséquents, des externalités positives4 et une baisse rapide des frais de transaction.
	
	Obsédés par leur quête de pouvoir, les GAFAM ont progressivement envahi la vie de leurs utilisateurs, allant parfois jusqu'à leur imposer leurs propres choix. Cette intrusion s'est déroulée de manière méthodique et calculée comme suit:
	\begin{itemize}
		\item Concentration du Pouvoir ;
		\item Protection de la Vie Privée ;
		\item Influence sur l'Opinion Publique ;
		\item Régulation et Législation ;
		\item Censure et Liberté d'Expression;
		\item Dépendance Économique;
		
	\end{itemize}
	

	Les \textbf{GAFAM}, détiennent une emprise considérable sur l'espace public numérique, soulevant des enjeux politiques majeurs. Parmi ces enjeux, la concentration du pouvoir émerge comme une préoccupation centrale, impactant directement la gouvernance et la prise de décision politique.
	
		\begin{table}[h]
		
		\begin{tabular}{|l|l|l|l|}
			\hline
			\textbf{Entreprises} & \textbf{confiance}&\textbf{pas confiance}&\textbf{Sans opinion} \\
			\hline
			google & 47 & 48 & 4 \\
			\hline
			Amazon & 40 & 53 & 7 \\
			\hline
			Facebook & 72 & 20 & 8 \\
			\hline
			Apple & 40 & 44 & 8 \\
			\hline
			Microsoft & 42 & 43 & 15 \\
			\hline
			
		\end{tabular}
		
		\caption{\textbf{Données personnelles : niveau de confiance des internautes américains envers les GAFAM (en pourcentage)}}
		\label{}
		
	\end{table}
	
	\section{\underline{Concentration du Pouvoir}}
	
	Les \textbf{GAFAM} , détiennent une emprise considérable sur l'espace public numérique, soulevant des enjeux politiques majeurs. Parmi ces enjeux, la concentration du pouvoir émerge comme une préoccupation centrale, impactant directement la gouvernance et la prise de décision politique.
	Les GAFAM ont atteint un niveau de monopole inédit dans divers secteurs, consolidant ainsi une influence démesurée. Par exemple,\textbf{Google} domine le marché de la recherche en ligne, \textbf{Facebook} règne sur les médias sociaux, \textbf{Amazon} contrôle le commerce électronique, et \textbf{Microsoft} est prépondérant dans le domaine du logiciel. Cette concentration monopolistique soulève des inquiétudes politiques quant à la capacité de ces entreprises à dicter les normes et à orienter les politiques publiques en fonction de leurs intérêts particuliers.
	Les géant de la Silicon Valley\footnote[6]{le pôle des industries de pointe situé dans la partie sud-est de la région de la baie de San Francisco dans l'État de Californie}  exercent également leur pouvoir politique à travers des activités de lobbying \footnote[7]{activité d'influence ou de pression sur le pouvoir politique.} intensives. Leur capacité financière leur permet d'influencer les décisions politiques, de façonner les lois et régulations à leur avantage, et de contourner les initiatives visant à restreindre leur domination. Cette interférence politique suscite des débats sur la transparence, l'éthique, et soulève la question fondamentale de la séparation entre le pouvoir politique et le pouvoir économique.
	La concentration du pouvoir entre les mains des GAFAM soulève des questions fondamentales sur l'équité \textit{démocratique}. En tant que gardiens de vastes portions de l'espace public numérique, ces entreprises détiennent le pouvoir de façonner l'accès à l'information, influençant ainsi les opinions publiques. Les risques liés à la manipulation de l'information et à la polarisation politique posent des défis majeurs pour la démocratie et mettent en lumière la nécessité d'une régulation politique appropriée.
	la \textbf{concentration du pouvoir} des GAFAM représente un enjeu politique crucial, avec des implications profondes pour la démocratie, la souveraineté nationale et l'équité économique. L'élaboration de politiques publiques efficaces et la régulation adéquate deviennent impératives pour garantir que ces acteurs majeurs de l'espace public numérique ne deviennent pas des arbitres incontrôlés de la sphère politique. La préservation des principes démocratiques nécessite une action politique concertée visant à équilibrer le pouvoir et à assurer la transparence dans le paysage numérique contemporain.
	
	
	\section{\underline{Protection de la Vie Privée}}
	
	Dans l'ère numérique dominée par les GAFAM, la manière dont ces géants technologiques utilisent nos données personnelles soulève des préoccupations majeures quant à l'impact sur nos choix politiques. Les GAFAM, par le biais de leurs plateformes diversifiées, collectent et analysent massivement des données individuelles, créant ainsi des profils détaillés de leurs utilisateurs. Cette exploitation des données personnelles sert de base à des mécanismes sophistiqués de ciblage, visant à influencer les opinions politiques des utilisateurs.
	
	Premièrement, les algorithmes\footnote[8]{description d'une suite d'étapes permettant d'obtenir un résultat à partir d'éléments fournis en entrée.} de recommandation des GAFAM jouent un rôle crucial dans la manière dont l'information politique est présentée à chaque utilisateur. En analysant les habitudes de navigation, les interactions et les préférences, ces algorithmes façonnent un écosystème d'information personnalisé qui peut renforcer les convictions existantes ou exposer l'utilisateur à des contenus polarisants. Ainsi, les GAFAM peuvent créer des bulles informationnelles qui limitent la diversité des opinions et contribuent à la polarisation politique.
	
	Deuxièmement, la publicité ciblée basée sur les données personnelles permet aux GAFAM d'influencer directement les choix politiques en personnalisant les messages en fonction des profils individuels. En utilisant des données telles que l'emplacement, l'âge, les préférences et les historiques de recherche, ces entreprises peuvent diffuser des annonces politiques spécifiquement conçues pour susciter des réactions émotionnelles ou confirmer des préjugés, impactant ainsi les décisions politiques des utilisateurs.
	
	Cette exploitation des données personnelles par les GAFAM soulève des questions cruciales en matière de vie privée et de démocratie. D'une part, elle remet en question le consentement éclairé des utilisateurs quant à l'utilisation de leurs données. D'autre part, elle alimente le débat sur la nécessité d'une régulation plus stricte pour protéger les individus contre une manipulation politique insidieuse.
	
	L'utilisation des données personnelles par les GAFAM pour influencer nos choix politiques souligne l'importance critique d'un examen attentif et d'une régulation efficace. La nécessité de trouver un équilibre entre l'innovation technologique et la protection des droits individuels est plus pressante que jamais, afin de garantir une sphère politique numérique éthique, transparente et respectueuse des principes démocratiques fondamentaux.
	
	\section{\underline{Influence sur l'Opinion Publique}}
	 
	 les Géants de la tech (Google, Apple, Facebook, Amazon, Microsoft) occupent une place prépondérante dans la formation de l'opinion publique, exerçant une influence significative sur la manière dont les individus perçoivent le monde qui les entoure. Cette section explore le rôle central joué par les GAFAM dans la \textit{création} et la \textit{modulation} de l'opinion publique, mettant en lumière les mécanismes par lesquels ces géants technologiques façonnent le  perspectives individuelles et collectives.
	
	Tout d’abord, les \textbf{algorithmes de recommandation} des GAFAM jouent un rôle déterminant dans la manière dont l'information est présentée aux utilisateurs. En analysant les comportements en ligne, ces algorithmes dirigent l'attention des individus vers des contenus qui correspondent à leurs préférences préexistantes, créant ainsi des filtres de réalité qui peuvent renforcer les croyances existantes et générer des bulles informationnelles. Cette personnalisation de l'expérience en ligne influence directement la perception que chacun a des événements, des idées politiques et des groupes sociaux.
	
	Ensuite, \textbf{les plateformes sociales} des GAFAM jouent un rôle central dans la diffusion de l'information et la création de tendances. Les contenus viraux et les messages partagés massivement peuvent rapidement façonner l'agenda public et influencer les débats politiques. Les GAFAM deviennent ainsi des amplificateurs puissants, déterminant souvent quelles voix et quelles perspectives gagnent en visibilité, ce qui peut avoir des répercussions profondes sur la formation de l'opinion publique.
	
	Enfin, \textbf{la publicité ciblée}permet aux GAFAM d'influencer de manière plus directe l'opinion publique en diffusant des messages personnalisés. En utilisant des \textit{ données comportementales, démographiques et géographiques}, ces entreprises peuvent adapter leurs campagnes publicitaires pour influencer les attitudes, les opinions et même les comportements des utilisateurs. Cela soulève des questions éthiques sur la manipulation de l'opinion publique à des fins politiques ou commerciales.
	
	En conclusion, les GAFAM exercent une influence considérable sur l'opinion publique à travers des mécanismes sophistiqués de personnalisation, de diffusion et de ciblage. Bien que ces entreprises jouent un rôle essentiel dans la démocratisation de l'information, leur pouvoir croissant nécessite une réflexion approfondie sur les implications politiques et sociales de leur influence. La régulation et la transparence deviennent impératives pour garantir que l'opinion publique reste le résultat d'un processus démocratique équitable, préservant ainsi l'intégrité du débat public.
	
	\section{\underline{Régulation et Législation} }
	
	Les \textbf{gouvernements du monde} entier cherchent à réguler l'activité des GAFAM pour garantir une concurrence équitable, protéger la vie privée et assurer la sécurité en ligne. Les débats sur la régulation et la législation sont au cœur des enjeux politiques liés à ces entreprises.
	
	\section{\underline{Censure et Liberté d'Expression}}
	
	L'essor des GAFAM (Google, Apple, Facebook, Amazon, Microsoft) dans le paysage numérique contemporain soulève des questions cruciales sur la \textbf{censure} et la \textbf{\underline{liberté d'expression}}, remettant en question l'équilibre délicat entre la nécessité de réguler le discours en ligne et le \underline{respect fondamental de la liberté individuelle d'expression} . Cette partie explore l'impact politique des GAFAM dans ce contexte complexe.
	
	Premièrement, les GAFAM agissent en tant que gardiens majeurs de l'espace public numérique, détenant un pouvoir significatif pour modérer et censurer le contenu en ligne. Bien que ces entreprises justifient souvent ces actions au nom de la suppression de contenus haineux, trompeurs ou nuisibles, cela soulève des inquiétudes quant à la concentration du pouvoir de décision entre les mains d'entités privées. Les décisions de censure peuvent avoir des implications politiques majeures, façonnant la manière dont les événements sont interprétés et influençant la diversité des opinions disponibles.
	
	Deuxièmement, les GAFAM sont souvent confrontés à des critiques pour la manière dont ils gèrent la liberté d'expression sur leurs plateformes \url{https://fr.search.yahoo.com/search?fr=mcafee&type=E211FR1140G0&p=gafam}. Les choix de modération peuvent être perçus comme biaisés politiquement, suscitant des débats sur la neutralité des plateformes et la possibilité de censure sélective. Ces entreprises font face à la difficile tâche de trouver un équilibre entre la protection contre les discours nocifs et la préservation de la diversité des opinions, ce qui souligne le besoin d'une réflexion politique approfondie sur la manière de définir et de maintenir des normes de modération équitables.
	
	Troisièmement, les GAFAM opèrent dans un contexte mondial diversifié avec des lois et des normes de liberté d'expression variées. Leurs politiques de modération doivent naviguer entre les attentes culturelles et les exigences légales, soulevant des questions sur la manière dont ces entreprises peuvent assurer une cohérence tout en respectant les contextes locaux. Cette complexité accentue le caractère politique de la modération en ligne et souligne la nécessité de normes internationales claires et équitables.
	
	L'impact politique des GAFAM sur la censure et la liberté d'expression souligne l'urgence de repenser la manière dont les plateformes numériques gèrent le discours en ligne. La régulation, la transparence et le dialogue entre les parties prenantes deviennent essentiels pour garantir que la modération en ligne respecte les valeurs fondamentales de la démocratie tout en protégeant contre les abus potentiels. La censure et la liberté d'expression demeurent ainsi au cœur des débats sur la gouvernance de l'espace public numérique.
	
	\section{\underline{Dépendance Économique}}
	 
	 Google, Apple, Facebook, Amazon et Microsoft ont évolué pour devenir des acteurs incontournables de l'économie mondiale. Leur influence s'étend au-delà des frontières nationales, suscitant des préoccupations quant à l'impact politique sur la dépendance économique des pays. Cette partie examinera comment les activités des GAFAM peuvent influencer la souveraineté économique des nations.
	 
	 Les GAFAM ont réussi à étendre leur emprise en capitalisant sur l'économie numérique mondiale. Dans son livre "World Without Mind", de Franklin Foer \cite{Foer2018} met en évidence comment ces entreprises ont réussi à façonner nos pensées et nos comportements en contrôlant les flux d'informations. Cette influence démesurée sur la diffusion des idées peut avoir des conséquences politiques importantes, sapant la diversité des perspectives et renforçant une vision biaisée du monde.
	 
	 En outre, la domination des GAFAM dans le secteur technologique peut conduire à une dépendance économique accrue des pays. Jean Tirole \footnote[9]{lauréat du prix Nobel d'économie}, souligne dans ses écrits que cette dépendance peut affaiblir la capacité des gouvernements à réguler efficacement ces entreprises transnationales. Les politiques publiques peuvent être contournées ou influencées par des acteurs privés, remettant en question la souveraineté des nations sur leur propre économie.
	 
	 Le livre "The Age of Surveillance Capitalism" de Shoshana Zuboff \cite{Zuboff2020} explore également comment les GAFAM utilisent les données personnelles comme une nouvelle forme de capital, entraînant une asymétrie de pouvoir entre ces entreprises et les gouvernements. Cette concentration de pouvoir économique peut rendre les nations vulnérables aux pressions et aux manipulations des GAFAM, affectant ainsi leur autonomie politique.
	 
	 Cependant, certains chercheurs soulignent que les GAFAM ne sont pas uniquement une menace, mais peuvent aussi être des partenaires potentiels pour les États. Dans "The Big Nine" de Amy Webb \cite{Webb2019} avance l'idée que la collaboration entre les entreprises technologiques et les gouvernements peut stimuler l'innovation et renforcer la compétitivité mondiale. Cependant, cela nécessite une régulation adéquate pour prévenir les abus de pouvoir.
	 
	 En définitive, on peut dire que l'impact politique des GAFAM sur la dépendance économique des pays est complexe et multifacette. Alors que ces entreprises peuvent apporter des avantages en termes d'innovation, elles soulèvent également des préoccupations sérieuses quant à la souveraineté économique et politique des nations. La nécessité de trouver un équilibre entre la collaboration et la régulation devient essentielle pour assurer un avenir où les GAFAM contribuent positivement au développement mondial sans compromettre la souveraineté des États.
	 
	
	\section{Conclusion}
	\section{Conclusion}
	%% bibliographie
	\printbibliography
	
\end{document}